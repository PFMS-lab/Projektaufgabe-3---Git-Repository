\documentclass[a4paper,12pt]{article}

\usepackage{svg}  % hinzufügen
\usepackage[utf8]{inputenc} 
\usepackage[T1]{fontenc} 
\usepackage{graphicx}
\usepackage{hyperref} % Für klickbare Links im Inhaltsverzeichnis
\usepackage{caption}  % Für bessere Bildunterschriften
\usepackage{geometry} % Seitenränder
\geometry{left=2.5cm, right=2.5cm, top=2.5cm, bottom=2.5cm}

\title{Werkzeuge für das wissenschaftliche Arbeiten\\
\large Python for Machine Learning and Data Science}
\author{}
\date{Abgabe: 15.12.2023}

\begin{document}

\maketitle
\hrule
\vspace{0.5cm}

\tableofcontents
\hrule
\vspace{1cm}

\section{Projektaufgabe}

In dieser Aufgabe beschäftigen wir uns mit Objektorientierung in Python.
Der Fokus liegt auf der Implementierung einer Klasse, dabei nutzen wir insbesondere auch Magic Methods.

\begin{figure}[h!]
    \centering
    \includegraphics[width=0.8\textwidth]{../diagram/image.png}
    \caption{Darstellung der Klassenbeziehungen.}
    \label{fig:classes}
\end{figure}

\subsection{Einleitung}

Ein Datensatz besteht aus mehreren Daten, ein einzelnes Datum wird durch ein Objekt der Klasse \texttt{DataSetItem} repräsentiert.
Jedes Datum hat einen Namen (Zeichenkette), eine ID (Zahl) und beliebigen Inhalt.

Mehrere Datenobjekte vom Typ \texttt{DataSetItem} sollen in einem Datensatz zusammengefasst werden.
Es gibt eine Klasse \texttt{DataSetInterface}, die die Schnittstelle definiert und Operationen jedes Datensatzes angibt.
Bisher fehlt aber noch die Implementierung eines Datensatzes mit allen Operationen.

Implementieren Sie eine Klasse \texttt{DataSet} als Unterklasse von \texttt{DataSetInterface}.

\subsection{Aufbau}

Es gibt drei Dateien: \texttt{dataset.py}, \texttt{main.py} und \texttt{implementation.py}.

\begin{itemize}
    \item In \texttt{dataset.py} befinden sich die Klassen \texttt{DataSetInterface} und \texttt{DataSetItem}.
    \item In \texttt{implementation.py} muss die Klasse \texttt{DataSet} implementiert werden.
    \item \texttt{main.py} nutzt die Klassen \texttt{DataSet} und \texttt{DataSetItem} und testet die Schnittstelle und Operationen.
\end{itemize}

\subsection{Methoden}

Folgende Methoden sind in der Klasse \texttt{DataSet} zu implementieren (genaue Spezifikation in \texttt{dataset.py}):

\begin{itemize}
    \item \_\_setitem\_\_(self, name, id\_content) -- Hinzufügen eines Datums.
    \item \_\_iadd\_\_(self, item) -- Hinzufügen eines \texttt{DataSetItem}.
    \item \_\_delitem\_\_(self, name) -- Löschen eines Datums nach Name.
    \item \_\_contains\_\_(self, name) -- Prüfen, ob ein Datum vorhanden ist.
    \item \_\_getitem\_\_(self, name) -- Abrufen eines Datums über Name.
    \item \_\_and\_\_(self, dataset) -- Schnittmenge zweier Datensätze zurückgeben.
    \item \_\_or\_\_(self, dataset) -- Vereinigungen zweier Datensätze zurückgeben.
    \item \_\_iter\_\_(self) -- Iteration über alle Daten.
    \item filtered\_iterate(self, filter) -- Gefilterte Iteration mittels Lambda-Funktion.
    \item \_\_len\_\_(self) -- Anzahl der Daten abrufen.
\end{itemize}

\section{Abgabe}

Programmieren Sie die Klasse \texttt{DataSet} in der Datei \texttt{implementation.py} zur Lösung der oben beschriebenen Aufgabe.  
Sie können auch direkt auf Ihrem Computer programmieren. Das VPL nutzt denselben Code, wobei \texttt{main.py} zusätzliche Testfälle enthält.

\end{document}
